De acordo com a NBR 6028 (ABNT, 2003), é a apresentação concisa dos pontos
relevantes de um texto, dando uma visão rápida e clara do conteúdo e das conclusões do trabalho, disposto antes das listas de ilustrações, abreviaturas, símbolos e o sumário. Para elaboração do resumo, devem-se seguir as seguintes orientações:

\begin{itemize}
	\item[(a)] deve ser informativo, apresentando finalidades, metodologia, resultados e conclusões;
	
	\item[(b)] composto de uma sequência de frases concisas, afirmativas e não de enumeração de tópicos;
	
	\item[(c)] usar o verbo na voz ativa e na 3ª pessoa do singular;
	
	\item[(d)] em trabalhos acadêmicos (teses, dissertações e outros) deve conter de 150 a 500 palavras;
	
	\item[(e)] a primeira frase do resumo deve ser significativa e expressar o tema principal do trabalho;
	
	\item[(f)] deve ser evitado o uso de frases negativas, símbolos e fórmulas que não sejam de uso corrente, comentário pessoal, críticas ou julgamento de valor; e,
	
	\item[(g)] as palavras-chave devem figurar abaixo do resumo, antecedidas da expressão \textbf{``Palavras-chave:''} separadas por ponto e vírgula e finalizadas por ponto;
	
	\item[(h)] evitar expressões como ``O presente trabalho...'', ``O autor descreve...''.
\end{itemize}

% Separe as palavras-chave por ponto e vírgula ';' e finalizadas por ponto.
\palavraschave{Sempre; olhe; o guia ; de normalização.}
