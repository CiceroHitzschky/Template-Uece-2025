Apresenta-se logo após o resumo em língua vernácula e em formato idêntico. É a tradução do mesmo para outro idioma de propagação internacional. Pode ser em inglês \textbf{(ABSTRACT)} em espanhol \textbf{(RESUMEN)} ou em francês \textbf{(RÉSUMÉ).} Tais palavras devem aparecer em letra maiúscula, negritada e centralizada na margem superior do trabalho e sem indicativo numérico e sem pontuação. As palavras-chave \textit{e/ou} descritores também devem ser traduzidas de acordo com o(s) idioma(s) escolhido(s), tais como: \textbf{Keywords} \textit{e/ou} \textbf{Descriptors} (inglês), \textbf{Palabras clave} \textit{e/ou} \textbf{Descriptores} (espanhol), \textbf{Mots-clés} \textit{e/ou} \textbf{Descripteurs} (francês). Todo o texto deve ser redigitado em fonte ARIAL ou \textit{TIMES NEW ROMAN}, um único parágrafo, justificado, tamanho da fonte 12, com espaçamento de 1,5 entrelinhas. Ver

% Separe as Keywords por ponto e vírgula ';' e finalize por .
\keywords{Always;  check the; standardization; Guide.}
